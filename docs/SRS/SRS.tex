% THIS DOCUMENT IS TAILORED TO REQUIREMENTS FOR SCIENTIFIC COMPUTING.  IT SHOULDN'T
% BE USED FOR NON-SCIENTIFIC COMPUTING PROJECTS
\documentclass[12pt]{article}

\usepackage{amsmath, mathtools}
\usepackage{amsfonts}
\usepackage{amssymb}
\usepackage{graphicx}
\usepackage{colortbl}
\usepackage{xr}
\usepackage{hyperref}
\usepackage{longtable}
\usepackage{xfrac}
\usepackage{tabularx}
\usepackage{float}
\usepackage{siunitx}
\DeclareSIUnit\pixel{pixel}
\usepackage{booktabs}
\usepackage{caption}
\usepackage{pdflscape}
\usepackage{afterpage}

\usepackage[round]{natbib}

%\usepackage{refcheck}

\hypersetup{
    bookmarks=true,         % show bookmarks bar?
      colorlinks=true,       % false: boxed links; true: colored links
    linkcolor=red,          % color of internal links (change box color with linkbordercolor)
    citecolor=green,        % color of links to bibliography
    filecolor=magenta,      % color of file links
    urlcolor=cyan           % color of external links
}

\input{../Comments.text}
\input{../Common.text}

% For easy change of table widths
\newcommand{\colZwidth}{1.0\textwidth}
\newcommand{\colAwidth}{0.13\textwidth}
\newcommand{\colBwidth}{0.82\textwidth}
\newcommand{\colCwidth}{0.1\textwidth}
\newcommand{\colDwidth}{0.05\textwidth}
\newcommand{\colEwidth}{0.8\textwidth}
\newcommand{\colFwidth}{0.17\textwidth}
\newcommand{\colGwidth}{0.5\textwidth}
\newcommand{\colHwidth}{0.28\textwidth}

% Used so that cross-references have a meaningful prefix
\newcounter{defnum} %Definition Number
\newcommand{\dthedefnum}{GD\thedefnum}
\newcommand{\dref}[1]{GD\ref{#1}}
\newcounter{datadefnum} %Datadefinition Number
\newcommand{\ddthedatadefnum}{DD\thedatadefnum}
\newcommand{\ddref}[1]{DD\ref{#1}}
\newcounter{theorynum} %Theory Number
\newcommand{\tthetheorynum}{TM\thetheorynum}
\newcommand{\tref}[1]{TM\ref{#1}}
\newcounter{tablenum} %Table Number
\newcommand{\tbthetablenum}{TB\thetablenum}
\newcommand{\tbref}[1]{TB\ref{#1}}
\newcounter{assumpnum} %Assumption Number
\newcommand{\atheassumpnum}{A\theassumpnum}
\newcommand{\aref}[1]{A\ref{#1}}
\newcounter{goalnum} %Goal Number
\newcommand{\gthegoalnum}{GS\thegoalnum}
\newcommand{\gsref}[1]{GS\ref{#1}}
\newcounter{instnum} %Instance Number
\newcommand{\itheinstnum}{IM\theinstnum}
\newcommand{\iref}[1]{IM\ref{#1}}
\newcounter{reqnum} %Requirement Number
\newcommand{\rthereqnum}{R\thereqnum}
\newcommand{\rref}[1]{R\ref{#1}}
\newcounter{nfrnum} %NFR Number
\newcommand{\rthenfrnum}{NFR\thenfrnum}
\newcommand{\nfrref}[1]{NFR\ref{#1}}
\newcounter{lcnum} %Likely change number
\newcommand{\lthelcnum}{LC\thelcnum}
\newcommand{\lcref}[1]{LC\ref{#1}}

\usepackage{fullpage}

\newcommand{\deftheory}[9][Not Applicable]
{
\newpage
\noindent \rule{\textwidth}{0.5mm}

\paragraph{RefName: } \textbf{#2} \phantomsection 
\label{#2}

\paragraph{Label:} #3

\noindent \rule{\textwidth}{0.5mm}

\paragraph{Equation:}

#4

\paragraph{Description:}

#5

\paragraph{Notes:}

#6

\paragraph{Source:}

#7

\paragraph{Ref.\ By:}

#8

\paragraph{Preconditions for \hyperref[#2]{#2}:}
\label{#2_precond}

#9

\paragraph{Derivation for \hyperref[#2]{#2}:}
\label{#2_deriv}

#1

\noindent \rule{\textwidth}{0.5mm}

}

\begin{document}

\title{Software Requirements Specification for \progname: Real-Time Exercise Form Analysis System} 
\author{\authname}
\date{\today}
	
\maketitle

~\newpage

\pagenumbering{roman}

\tableofcontents

~\newpage

\section*{Revision History}

\begin{tabularx}{\textwidth}{p{3cm}p{2cm}X}
\toprule {\bf Date} & {\bf Version} & {\bf Notes}\\
\midrule
2026-01-30 & 1.0 & First Draft for CAS 741\\
\bottomrule
\end{tabularx}

~\\

~\newpage

\section{Reference Material}

This section records information for easy reference.

\subsection{Table of Units}

Throughout this document SI (Syst\`{e}me International d'Unit\'{e}s) is employed
as the unit system.  In addition to the basic units, several derived units are
used as described below.  For each unit, the symbol is given followed by a
description of the unit and the SI name.
~\newline

\renewcommand{\arraystretch}{1.2}
%\begin{table}[ht]
  \noindent \begin{tabular}{l l l} 
    \toprule		
    \textbf{symbol} & \textbf{unit} & \textbf{SI}\\
    \midrule 
    \si{\degree} & angle & degree (derived)\\
    \si{\second} & time & second\\
    \si{\pixel} & length & pixel (screen coordinates)\\
    \bottomrule
  \end{tabular}
  %	\caption{Provide a caption}
%\end{table}

\subsection{Table of Symbols}

The table that follows summarizes the symbols used in this document along with
The choice of symbols was made to be consistent with the biomechanics and computer vision literature. The symbols are listed in alphabetical order.

\renewcommand{\arraystretch}{1.2}
%\noindent \begin{tabularx}{1.0\textwidth}{l l X}
\noindent \begin{longtable*}{l l p{12cm}} \toprule
\textbf{symbol} & \textbf{unit} & \textbf{description}\\
\midrule 
$\theta$ & \si{\degree} & Interior angle of a joint\\
$P$ & \si{\pixel} & A point in 2D space $(x, y)$\\
$\vec{v}$ & - & Vector connecting two keypoints\\
$Is\_Valid$ & - & Boolean flag for repetition validity\\
\bottomrule
\end{longtable*}

\subsection{Abbreviations and Acronyms}

\renewcommand{\arraystretch}{1.2}
\begin{tabular}{l l} 
  \toprule		
  \textbf{symbol} & \textbf{description}\\
  \midrule 
  A & Assumption\\
  DD & Data Definition\\
  GD & General Definition\\
  GS & Goal Statement\\
  IM & Instance Model\\
  LC & Likely Change\\
  PS & Physical System Description\\
  R & Requirement\\
  SRS & Software Requirements Specification\\
  \progname{} & FitCoachAR\\
  TM & Theoretical Model\\
  RGB & Red Green Blue (Video format)\\
  \bottomrule
\end{tabular}\\



\newpage

\pagenumbering{arabic}

\section{Introduction}

\subsection{Purpose of Document}
The purpose of this document is to specify the functional and non-functional requirements for \progname{}. This document is intended to be used by the development team for implementation and verification, and by the stakeholders to ensure the system meets the biomechanical analysis needs.

\subsection{Scope of Requirements} 
The scope of \progname{} is limited to the \textbf{2D analysis} of human movement using a single camera. The system currently focuses on the \textbf{Squat} and \textbf{Bicep Curl} exercises. It ignores 3D depth perception (z-axis) unless inferred, and assumes a stable camera position. The system calculates geometric angles based on projected keypoints and does not measure force, torque, or muscle activity.

\subsection{Characteristics of Intended Reader} \label{sec_IntendedReader}
The intended reader is expected to have knowledge of:
\begin{itemize}
    \item Basic biomechanics (joint angles, flexion, extension).
    \item Software Engineering principles (SRS structure).
    \item Basic Linear Algebra (vectors, dot products).
\end{itemize}

\subsection{Organization of Document}
This document follows the Scientific Computing SRS template. It starts with the System Description, followed by the specific physics/math models (TMs, DDs, IMs), and concludes with the Requirements.
  

\section{General System Description}

\progname{} is a real-time fitness coaching application that uses computer vision to analyze human movement during exercises. By leveraging a standard webcam, the system detects the user's body pose, calculates joint angles, and provides immediate feedback on exercise form. The goal is to help users perform exercises correctly, reducing injury risk and improving effectiveness, without requiring expensive equipment like motion capture suits.

\subsection{System Context}
\progname{} operates as a local application. The user interacts directly with the software via a webcam and a display screen.
\begin{itemize}
    \item \textbf{User Responsibilities:} Ensure good lighting, wear distinguishable clothing, and position themselves fully within the camera frame.
    \item \textbf{\progname{} Responsibilities:} Detect the user, track keypoints, calculate angles, and provide feedback in real-time.
\end{itemize}

\begin{figure}[h!]
\begin{center}
 \includegraphics[width=0.9\textwidth]{SystemContextFigure.png}
\caption{System Context: User $\to$ Webcam $\to$ FitCoachAR $\to$ Screen}
\label{Fig_SystemContext} 
\end{center}
\end{figure}

\subsection{User Characteristics} \label{SecUserCharacteristics}
The end user is expected to be a fitness enthusiast or a patient undergoing physical therapy. They are assumed to have basic computer literacy but no technical knowledge of computer vision algorithms.

\subsection{System Constraints}
\begin{itemize}
    \item The system is deployed as a \textbf{web application}, accessible via modern browsers (Chrome, Firefox, Safari, Edge) on desktop and mobile devices.
    \item The system depends on a \textbf{pose estimation backend} (e.g., MediaPipe or MoveNet) for raw keypoint extraction. The specific library may be swapped without affecting core application logic.
    \item The user's device must have a camera capable of capturing video at a minimum of 720p resolution.
\end{itemize}

\section{Specific System Description}

\subsection{Problem Description} \label{Sec_pd}
Improper form during exercises like Squats can lead to injury and reduced effectiveness. \progname{} aims to solve this by providing automated, real-time coaching feedback without the need for expensive motion capture suits.

\subsubsection{Physical System Description} \label{sec_phySystDescrip}
The physical system includes:
\begin{itemize}
\item[PS1:] The \textbf{User}, whose body joints (Hip, Knee, Ankle for legs; Shoulder, Elbow, Wrist for arms) are the targets of measurement.
\item[PS2:] The \textbf{Video Stream}, which captures the 2D projection of the User's movement.
\end{itemize}

\subsubsection{Goal Statements}
\begin{itemize}
\item[GS\refstepcounter{goalnum}\thegoalnum \label{G_Landmarks}:] Analyze user video input to detect skeletal landmarks (Legs and Arms) in 2D space ($\mathbb{R}^2$).
\item[GS\refstepcounter{goalnum}\thegoalnum \label{G_Verify}:] Verify movement execution against established biomechanical thresholds (e.g., knee interior angle, elbow extension).
\item[GS\refstepcounter{goalnum}\thegoalnum \label{G_Repetition}:] Quantify valid repetitions.
\end{itemize}

\subsection{Solution Characteristics Specification}

\subsubsection{Assumptions} \label{sec_assumpt}
\begin{itemize}
\item[A\refstepcounter{assumpnum}\theassumpnum \label{A_2D}:] The 2D projection of the user's side profile is sufficient to approximate the true biomechanical angles of the knee for a Squat.
\item[A\refstepcounter{assumpnum}\theassumpnum \label{A_Camera}:] The camera remains stationary during the exercise.
\item[A\refstepcounter{assumpnum}\theassumpnum \label{A_Visibility}:] The relevant joints (Hip, Knee, Ankle for Squats; Shoulder, Elbow, Wrist for Bicep Curls) are not occluded during the movement.
\end{itemize}

\noindent 





\subsubsection{Theoretical Models}\label{sec_theoretical}

This section focuses on the general equations and laws that \progname{} is based
on.

~\newline

\noindent
\deftheory
{TM:VectorAngle}
{Calculation of Interior Angle between Two Vectors}
{
  $\theta = \arccos \left( \frac{\vec{v}_1 \cdot \vec{v}_2}{|\vec{v}_1| |\vec{v}_2|} \right)$
}
{
  This model defines the interior angle $\theta$ between two vectors $\vec{v}_1$ and $\vec{v}_2$ in Euclidean space. 
  The dot product $\cdot$ and the magnitude $|.|$ are standard vector operations.
}
{None.}
{Standard Linear Algebra Textbooks}
{\dref{GD_KneeAngle}}
{Non-zero magnitude vectors.}

~\newline

\begin{minipage}{\textwidth}
\renewcommand*{\arraystretch}{1.5}
\begin{tabular}{| p{\colAwidth} | p{\colBwidth}|}
\hline
\rowcolor[gray]{0.9}
Number& DD\refstepcounter{datadefnum}\thedatadefnum \label{DD_Landmarks}\\
\hline
Label& \bf Skeletal Keypoints (Legs)\\
\hline
Symbol &$P_H, P_K, P_A$\\
\hline
SI Units & \si{\pixel} (x, y coordinates)\\
\hline
Description &
$P_H$: Coordinates of the Hip joint.
$P_K$: Coordinates of the Knee joint.
$P_A$: Coordinates of the Ankle joint.
These are provided by the pose estimation library (MediaPipe).
\\
\hline
Sources& MediaPipe Documentation \\
\hline
Ref.\ By & \dref{GD_KneeAngle}\\
\hline
\end{tabular}
\end{minipage}

\vspace{0.5cm}

\begin{minipage}{\textwidth}
\renewcommand*{\arraystretch}{1.5}
\begin{tabular}{| p{\colAwidth} | p{\colBwidth}|}
\hline
\rowcolor[gray]{0.9}
Number& DD\refstepcounter{datadefnum}\thedatadefnum \label{DD_Landmarks_Arm}\\
\hline
Label& \bf Skeletal Keypoints (Arms)\\
\hline
Symbol &$P_S, P_E, P_W$\\
\hline
SI Units & \si{\pixel} (x, y coordinates)\\
\hline
Description &
$P_S$: Coordinates of the Shoulder joint.
$P_E$: Coordinates of the Elbow joint.
$P_W$: Coordinates of the Wrist joint.
These are provided by the pose estimation backend.
\\
\hline
Sources& Standard Biomechanics \\
\hline
Ref.\ By & \dref{GD_ElbowAngle}\\
\hline
\end{tabular}
\end{minipage}

\subsubsection{General Definitions}\label{sec_gendef}

This section collects the laws and equations that will be used in building the
instance models.

~\newline

\noindent
\begin{minipage}{\textwidth}
\renewcommand*{\arraystretch}{1.5}
\begin{tabular}{| p{\colAwidth} | p{\colBwidth}|}
\hline
\rowcolor[gray]{0.9}
Number& GD\refstepcounter{defnum}\thedefnum \label{GD_KneeAngle}\\
\hline
Label &\bf Knee Joint Angle Calculation \\
\hline
SI Units&\si{\degree}\\
\hline
Equation& $\theta_{knee} = \text{TM:VectorAngle}(\vec{v}_{femur}, \vec{v}_{tibia})$  \\
\hline
Description &
The knee angle is calculated using \tref{TM:VectorAngle}. 
$\vec{v}_{femur}$ is the vector from Knee to Hip.
$\vec{v}_{tibia}$ is the vector from Knee to Ankle.
This assumes the 2D plane projection (\aref{A_2D}).
\\
\hline
Ref.\ By & \iref{IM_SquatValid}\\
  \hline
\end{tabular}
\end{minipage}

\vspace{0.5cm}

\begin{minipage}{\textwidth}
\renewcommand*{\arraystretch}{1.5}
\begin{tabular}{| p{\colAwidth} | p{\colBwidth}|}
\hline
\rowcolor[gray]{0.9}
Number& GD\refstepcounter{defnum}\thedefnum \label{GD_ElbowAngle}\\
\hline
Label &\bf Elbow Joint Angle Calculation \\
\hline
SI Units&\si{\degree}\\
\hline
Equation& $\theta_{elbow} = \text{TM:VectorAngle}(\vec{v}_{humerus}, \vec{v}_{radius})$  \\
\hline
Description &
The elbow angle is calculated using \tref{TM:VectorAngle}. 
$\vec{v}_{humerus}$ is the vector from Elbow to Shoulder.
$\vec{v}_{radius}$ is the vector from Elbow to Wrist.
This assumes non-occluded view (\aref{A_Visibility}).
\\
\hline
Ref.\ By & \iref{IM_BicepCurlValid}, \iref{IM_RepetitionCount}\\
  \hline
\end{tabular}
\end{minipage}

\vspace{0.5cm}

\begin{minipage}{\textwidth}
\renewcommand*{\arraystretch}{1.5}
\begin{tabular}{| p{\colAwidth} | p{\colBwidth}|}
\hline
\rowcolor[gray]{0.9}
Number& GD\refstepcounter{defnum}\thedefnum \label{GD_NormalizedProgress}\\
\hline
Label &\bf Normalized Repetition Progress \\
\hline
SI Units& Dimensionless ($0..1$)\\
\hline
Equation& $p = \frac{\theta - \theta_{low}}{\theta_{high} - \theta_{low}}$  \\
\hline
Description &
Maps the current angle $\theta$ to a progress value $p \in [0, 1]$.
$\theta_{low}$ and $\theta_{high}$ are calibrated ROM limits for the user.
If $p \approx 0$, the user is at the start (eccentric start).
If $p \approx 1$, the user is at the peak (concentric end).
\\
\hline
Ref.\ By & \iref{IM_RepetitionCount}\\
  \hline
\end{tabular}
\end{minipage}

\subsubsection{Instance Models} \label{sec_instance}

This section details the instance models that use the general definitions to provide the necessary solution.

~\newline

\noindent
\begin{minipage}{\textwidth}
\renewcommand*{\arraystretch}{1.5}
\begin{tabular}{| p{\colAwidth} | p{\colBwidth}|}
  \hline
  \rowcolor[gray]{0.9}
  Number& IM\refstepcounter{instnum}\theinstnum \label{IM_SquatValid}\\
  \hline
  Label& \bf Squat Depth Validity Criteria\\
  \hline
  Input& $\theta_{knee}$ (\si{\degree})\\
  \hline
  Output& $IsValid$ (Boolean)\\
  \hline
  Description& 
  $IsValid = (\theta_{knee} \geq \theta_{thresh})$, where $\theta_{thresh} = 90^\circ$ (or user defined).
  The squat is considered valid if the knee angle meets or exceeds the threshold.
  \\
  \hline
  Sources& Standard Squat Biomechanics \\
  \hline
  Ref.\ By & \rref{R_Verify}\\
  \hline
\end{tabular}
\end{minipage}

\vspace{0.5cm}

\begin{minipage}{\textwidth}
\renewcommand*{\arraystretch}{1.5}
\begin{tabular}{| p{\colAwidth} | p{\colBwidth}|}
  \hline
  \rowcolor[gray]{0.9}
  Number& IM\refstepcounter{instnum}\theinstnum \label{IM_BicepCurlValid}\\
  \hline
  Label& \bf Bicep Curl Validity Criteria\\
  \hline
  Input& $\theta_{elbow}$ (\si{\degree})\\
  \hline
  Output& $IsValid$ (Boolean)\\
  \hline
  Description& 
  $IsValid = (\theta_{elbow} \geq 160^\circ \text{ AND } \theta_{elbow} \leq 30^\circ)$.
  A repetition is valid if the arm starts at full extension ($>160^\circ$) and curls to full flexion ($<30^\circ$).
  \\
  \hline
  Sources& Standard Bicep Curl Biomechanics \\
  \hline
  Ref.\ By & \rref{R_Verify}\\
  \hline
\end{tabular}
\end{minipage}

\vspace{0.5cm}

\begin{minipage}{\textwidth}
\renewcommand*{\arraystretch}{1.5}
\begin{tabular}{| p{\colAwidth} | p{\colBwidth}|}
  \hline
  \rowcolor[gray]{0.9}
  Number& IM\refstepcounter{instnum}\theinstnum \label{IM_RepetitionCount}\\
  \hline
  Label& \bf Repetition Counting State Machine\\
  \hline
  Input& $p$ (from \dref{GD_NormalizedProgress}), $t$ (time)\\
  \hline
  Output& $RepCount$ ($\mathbb{N}$)\\
  \hline
  Description& 
  A full repetition is counted ($RepCount \leftarrow RepCount + 1$) when the state machine completes the cycle:
  1. \textbf{Start}: $p \leq 0.15$ (Bottom State)
  2. \textbf{Ascend}: $p$ increases $> 0.15$ (Up Phase)
  3. \textbf{Peak}: $p \geq 0.85$ (Top State)
  4. \textbf{Descend}: $p$ decreases $< 0.85$ (Down Phase)
  5. \textbf{Complete}: $p \leq 0.15$ (Return to Bottom)
  Constraints: The full cycle duration $\Delta t$ must satisfy $t_{min} \leq \Delta t \leq t_{max}$.
  \\
  \hline
  Sources& Discrete Event Systems, Signal Processing \\
  \hline
  Ref.\ By & \rref{R_Output}\\
  \hline
\end{tabular}
\end{minipage}

\section{Requirements}

\subsection{Functional Requirements}

\noindent \begin{itemize}

\item[R\refstepcounter{reqnum}\thereqnum \label{R_Input}:] The system shall accept real-time video input from a webcam.
\item[R\refstepcounter{reqnum}\thereqnum \label{R_Process}:] The system shall track skeletal keypoints for legs (\ddref{DD_Landmarks}) and arms (\ddref{DD_Landmarks_Arm}) in each frame.
\item[R\refstepcounter{reqnum}\thereqnum \label{R_Calc}:] The system shall calculate the knee angle $\theta_{knee}$ using \dref{GD_KneeAngle} and elbow angle $\theta_{elbow}$ using \dref{GD_ElbowAngle}.
\item[R\refstepcounter{reqnum}\thereqnum \label{R_Verify}:] The system shall determine the validity of the exercise based on \iref{IM_SquatValid} (for Squats) or \iref{IM_BicepCurlValid} (for Bicep Curls).
\item[R\refstepcounter{reqnum}\thereqnum \label{R_Output}:] The system shall display the skeletal overlay and feedback text on the screen.

\end{itemize}

\subsection{Nonfunctional Requirements}

\noindent \begin{itemize}

\item[NFR\refstepcounter{nfrnum}\thenfrnum \label{NFR_Accuracy}:] \textbf{Accuracy:} The calculated angles shall be within $\pm 5^\circ$ of visual ground truth for non-occluded frames.
\item[NFR\refstepcounter{nfrnum}\thenfrnum \label{NFR_Portability}:] \textbf{Portability:} The system shall be platform-independent, accessible on any device with a modern web browser and camera.


\end{itemize}

\subsection{Rationale}

The decisions captured in this document are based on the requirements of the CAS 741 course project and the goal of creating a scientifically rigorous exercise form analysis tool using 2D computer vision.

\section{Likely Changes}    
\begin{itemize}
\item[LC\refstepcounter{lcnum}\thelcnum:] The threshold for squat depth ($\theta_{thresh}$) and bicep curl range may become customizable by the user instead of hardcoded.
\item[LC\refstepcounter{lcnum}\thelcnum:] Additional exercises (e.g., Lunges, Shoulder Press) may be added, requiring new IMs.
\end{itemize}

\section{Traceability Matrices and Graphs}

The purpose of the traceability matrices is to provide easy references on what
has to be additionally modified if a certain component is changed.  Every time a
component is changed, the items in the column of that component that are marked
with an ``X'' may have to be modified as well.  Table~\ref{Table:trace} shows the
dependencies of theoretical models, general definitions, data definitions, and
instance models with each other. Table~\ref{Table:R_trace} shows the
dependencies of instance models, requirements, and data constraints on each
other. Table~\ref{Table:A_trace} shows the dependencies of theoretical models,
general definitions, data definitions, instance models, and likely changes on
the assumptions.

\begin{table}[h!]
\centering
\begin{tabular}{|c|c|c|c|}
\hline
	& \aref{A_2D}& \aref{A_Camera}& \aref{A_Visibility} \\
\hline
\dref{GD_KneeAngle}  & X & & \\ \hline
\dref{GD_ElbowAngle} & X & & X \\ \hline
\dref{GD_NormalizedProgress} & & & \\ \hline
\iref{IM_SquatValid} & X & & \\ \hline
\iref{IM_BicepCurlValid} & X & & \\ \hline
\iref{IM_RepetitionCount} & & & \\ \hline
\end{tabular}
\caption{Traceability Matrix Showing the Connections Between Assumptions and Other Items}
\label{Table:A_trace}
\end{table}

\begin{table}[h!]
\centering
\begin{tabular}{|c|c|c|c|c|c|c|}
\hline        
	& \tref{TM:VectorAngle}& \ddref{DD_Landmarks}& \ddref{DD_Landmarks_Arm} & \dref{GD_KneeAngle} & \dref{GD_ElbowAngle} & \dref{GD_NormalizedProgress} \\
\hline
\dref{GD_KneeAngle}  & X & X & & & & \\ \hline
\dref{GD_ElbowAngle} & X & & X & & & \\ \hline
\dref{GD_NormalizedProgress} & & & & & & \\ \hline
\iref{IM_SquatValid} & & & & X & & \\ \hline
\iref{IM_BicepCurlValid} & & & & & X & \\ \hline
\iref{IM_RepetitionCount} & & & & & & X \\ \hline
\end{tabular}
\caption{Traceability Matrix Showing the Connections Between Items of Different Sections}
\label{Table:trace}
\end{table}

\begin{table}[h!]
\centering
\begin{tabular}{|c|c|c|c|c|c|c|c|}
\hline
	& \ddref{DD_Landmarks}& \ddref{DD_Landmarks_Arm} & \dref{GD_KneeAngle}& \dref{GD_ElbowAngle} & \dref{GD_NormalizedProgress} & \iref{IM_SquatValid} & \iref{IM_BicepCurlValid} \\
\hline
\rref{R_Process}       & X & X & & & & & \\ \hline
\rref{R_Calc}          & & & X & X & X & & \\ \hline
\rref{R_Verify}        & & & & & & X & X \\ \hline
\rref{R_Output}        & & & & & & X & X \\ \hline
\end{tabular}
\caption{Traceability Matrix Showing the Connections Between Requirements and Instance Models}
\label{Table:R_trace}
\end{table}

The purpose of the traceability graphs is also to provide easy references on
what has to be additionally modified if a certain component is changed.  The
arrows in the graphs represent dependencies. The component at the tail of an
arrow is depended on by the component at the head of that arrow. Therefore, if a
component is changed, the components that it points to should also be
changed.

\section{Development Plan}

Metric-based verification will be prioritized in the first iteration.

\section{Values of Auxiliary Constants}

None.

\newpage

\bibliographystyle {plainnat}
\bibliography {../../refs/References}

\newpage

\newpage{}
\section*{Appendix --- Reflection}

\textit{(Not required for CAS 741 submission.)}
  

\input{../Reflection.text}

\input{../SRS_Reflection.text}

\end{document}