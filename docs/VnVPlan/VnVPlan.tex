\documentclass[12pt, titlepage]{article}

\usepackage{booktabs}
\usepackage{tabularx}
\usepackage{hyperref}
\hypersetup{
    colorlinks,
    citecolor=blue,
    filecolor=black,
    linkcolor=red,
    urlcolor=blue
}
\usepackage[round]{natbib}

\input{../Comments.text}
\input{../Common.text}

\begin{document}

\title{System Verification and Validation Plan for \progname{}} 
\author{\authname}
\date{\today}
        
\maketitle

\pagenumbering{roman}

\section*{Revision History}

\begin{tabularx}{\textwidth}{p{3cm}p{2cm}X}
\toprule {\bf Date} & {\bf Version} & {\bf Notes}\\
\midrule
Feb 12, 2026 & 1.0 & Initial Plan based on Presentation Strategy\\
\bottomrule
\end{tabularx}

~\newpage

\tableofcontents

\listoftables

\listoffigures

\newpage

\section{Symbols, Abbreviations, and Acronyms}

\renewcommand{\arraystretch}{1.2}
\begin{tabular}{l l} 
  \toprule              
  \textbf{symbol} & \textbf{description}\\
  \midrule 
  T & Test\\
  SRS & Software Requirements Specification\\
  VnV & Verification and Validation\\
  \bottomrule
\end{tabular}\\

\newpage

\pagenumbering{arabic}

This document outlines the Verification and Validation (VnV) plan for the \progname{} system.

\section{General Information}

\subsection{Summary}
\progname{} is a real-time exercise feedback system that uses computer vision (MediaPipe) to estimate pose and a state machine to count repetitions and analyze form.

\subsection{Objectives}
The primary objectives of this VnV plan are to:
\begin{itemize}
    \item Build confidence in the correctness of the \textbf{repetition counting logic}.
    \item Verify the \textbf{accuracy} of the biomechanical angle calculations.
    \item Ensure the system operates in \textbf{real-time}.
\end{itemize}

\subsection{Relevant Documentation}
\citet{SRS} provides the functional and non-functional requirements that will be verified.

\section{Plan}

\subsection{Verification and Validation Team}
The verification team consists of the primary developer (\authname) and the assigned peer reviewers.

\subsection{SRS Verification}
The SRS has been verified through:
\begin{itemize}
    \item Peer review by the Domain Expert (Yibing Mei) via GitHub Issues.
    \item Review by the supervisor (Dr. Smith).
    \item Checklist-based verification of Goal Statements and Input Constraints.
\end{itemize}

\subsection{Design Verification}
The Design (MG and MIS) will be verified through:
\begin{itemize}
    \item Peer review (GitHub Issues).
    \item Comparison against the SRS to ensure all requirements are mapped to modules.
\end{itemize}

\subsection{Implementation Verification}
Implementation verification will focus on \textbf{Manual Unit Testing} of the core logic and \textbf{Automated Testing} where feasible.
\begin{itemize}
    \item \textbf{Static Analysis}: Code reviews and linting.
    \item \textbf{Dynamic Testing}: Manual execution of logic test cases (see Section 4).
\end{itemize}

\subsection{Automated Testing and Verification Tools}
The following tools will be used:
\begin{itemize}
    \item \textbf{Pytest}: For future automated unit tests.
    \item \textbf{GitHub Actions}: For Continuous Integration (CI).
    \item \textbf{MediaPipe}: External library (assumed verified) for pose estimation.
\end{itemize}

\subsection{Software Validation}
Validation will compare the system's output against ground truth:
\begin{itemize}
    \item \textbf{Geometric Verification}: Comparing calculated angles against physical goniometer measurements (Ground Truth).
\end{itemize}

\section{System Tests}

\subsection{Tests for Functional Requirements}

\subsubsection{Repetition Counting (R4)}

\begin{enumerate}
\item{T1: Perfect Squat\\}
Type: Functional, Dynamic, Manual Simulation
Initial State: System Initialized
Input: Sequence of angles: $180^\circ \rightarrow 90^\circ \rightarrow 180^\circ$
Output: `RepCount` increments by 1. State transitions: Standing $\rightarrow$ Descending $\rightarrow$ Bottom $\rightarrow$ Ascending $\rightarrow$ Standing.
Test Case Derivation: Tests the happy path of a full valid repetition.
How test will be performed: Manually feed angle data to the state machine function.

\item{T2: Failed Rep\\}
Type: Functional, Dynamic, Manual Simulation
Initial State: System Initialized
Input: Sequence of angles: $180^\circ \rightarrow 100^\circ \rightarrow 180^\circ$
Output: `RepCount` does NOT increment. Feedback "Go Deeper" is triggered.
Test Case Derivation: Tests the threshold logic (Boundary Value Analysis).
How test will be performed: Manually feed angle data to the state machine function.
\end{enumerate}

\subsection{Tests for Nonfunctional Requirements}

\subsubsection{Accuracy (NFR1)}

\begin{enumerate}
\item{T3: Geometric Verification\\}
Type: Functional/Non-Functional, Dynamic, Manual
Initial State: System Initialized
Input: Static images with known joint angles (measured physically).
Output: Calculated angle within $\pm 5^\circ$ of proper measurement.
Test Case Derivation: Verification of the core measurement engine accuracy.
How test will be performed: Compare MediaPipe output with goniometer readings.
\end{enumerate}

\section{Unit Test Description}

\subsection{Unit Testing Scope}
Unit testing will focus on the \textbf{Analysis Module} (State Machine), as it contains the critical business logic. The UI and Camera modules are tested via System Testing.

\subsection{Tests for Functional Requirements}
The tests T1 and T2 defined in the System Tests section also serve as the primary Unit Tests for the Analysis Module.

\bibliographystyle{plainnat}
\bibliography{../../refs/References}

\newpage

\section{Appendix}
N/A

\newpage{}
\section*{Appendix --- Reflection}

The information in this section will be used to evaluate the team members on the
graduate attribute of Lifelong Learning.

\input{../Reflection.text}

\begin{enumerate}
  \item What went well? Planning the "Manual Unit Tests" helped clarify the logic before writing code.
  \item Pain points? Determining how to verify accuracy without an expensive motion capture system. Resolved by using static goniometer measurements.
  \item Skills needed? Knowledge of Python `pytest` for future automation and MediaPipe constraints.
\end{enumerate}

\end{document}