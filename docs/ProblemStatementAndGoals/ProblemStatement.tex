\documentclass{article}

\usepackage{tabularx}
\usepackage{booktabs}
\usepackage{hyperref}

\newcommand{\progname}{FitCoachAR}
\newcommand{\authname}{Seyedmohamad Mirhosseininejad}

\title{CAS 741: Problem Statement and Goals\\\progname}

\author{\authname}

\date{\today}

\input{../Comments.text}

\begin{document}

\maketitle

\begin{table}[hp]
\caption{Revision History} \label{TblRevisionHistory}
\begin{tabularx}{\textwidth}{llX}
\toprule
\textbf{Date} & \textbf{Developer(s)} & \textbf{Change}\\
\midrule
2026-01-17 & S. Mirhosseininejad & Initial Draft for CAS 741\\
\bottomrule
\end{tabularx}
\end{table}

\section{Problem Statement}

\subsection{Problem}
The \textbf{FitCoachAR} system, in its current implementation, is a research prototype which was developed for mobile data analytics course project \cite{1}. Although the core algorithms for the post-processing of pose estimation (such as Kalman filtering and geometric canonicalization) and the "FormScript" language are experimentally validated, the current software architecture is remaining monolithic. It lacks the rigorous software engineering qualities that are required for clinical or commercial deployment contexts.

To be specific, the system is suffering from the following issues:
\begin{itemize}
    \item \textbf{Missing Data Persistence:} There is not a structured database to store the longitudinal data of users. This means that calibration parameters and "FormCodes" are lost after each session, which prevents the tracking of long-term progress.
    \item \textbf{Configurations are Hardcoded:} The logic of exercises is coupled tightly with the application code. This makes it difficult for the domain experts (physiotherapists) to add new "Super FormCodes" without modification of the source code.
    \item \textbf{Validation is Insufficient:} Although the algorithms are working, the codebase is lacking a comprehensive suite of unit tests to guarantee that the geometric calculations are correct across the edge cases.
\end{itemize}

\subsection{Inputs and Outputs}
\begin{itemize}
    \item \textbf{Inputs:} Real-time video frames or the files of pre-recorded video; The calibration parameters specific to user (Range of Motion).
    \item \textbf{Outputs:}
    \begin{itemize}
        \item Feedback Systems:
        \begin{itemize}
            \item \textbf{Intra-Rep Guidance:} Real-time corrective feedback (e.g., "Go deeper") and AR directional arrows that are provided \textit{during} the phase of movement.
            \item \textbf{Post-Rep Summary:} Immediate, brief validation (e.g., "Nice rep" or "Less swing") which is triggered instantly upon the completion of repetition.
            \item \textbf{Post-Session Analysis:} A comprehensive report generated by LLM containing the overall "Form Score", classifications of frame-by-frame "FormCode", and the aggregated metrics of error.
        \end{itemize}
        \item Persistent records of the performance of user stored in a local database.
    \end{itemize}
\end{itemize}

\subsection{Stakeholders}
\begin{itemize}
    \item \textbf{Researchers/Developers:} Those who need a modular "FormScript" library for experimenting with the new algorithms of pose analysis.
    \item \textbf{Physiotherapists:} Those who need to define the new rules of exercises (FormCodes) without doing coding.
    \item \textbf{End Users:} Those who need to track their improvement of form over weeks, which requires the persistence of data.
\end{itemize}

\subsection{Environment}
The system is using Python for the logic of backend. Also, the core logic relies on \textbf{NumPy} for the calculations of vectors and \textbf{MediaPipe/MoveNet} for the data of pose. The new iteration will be introducing \textbf{SQLite/SQLAlchemy} for the persistence of data and Pytest for the verification purpose.

\section{Goals}
\begin{enumerate}
    \item \textbf{Modularization:} To refactor the widespread monolithic backend into three scientific computing modules that are decoupled and independently testable:
    \begin{itemize}
        \item \textbf{Pose Estimator:} It handles the interface with MoveNet/MediaPipe and also the pipeline of geometric post-processing.
        \item \textbf{Rep Counter:} It manages the state machine for the detection of cycles and the tracking of phases.
        \item \textbf{Feedback Engine:} It executes the logic of "FormScript" to generate the guidance of intra-rep and summaries of post-rep.
    \end{itemize}
    \item \textbf{Data Persistence:} To integrate a relational database (SQLite) for storing the profiles of users, thresholds of calibration, and the historical "Form Scores," which currently are lost after every session.
    \item \textbf{Validation:} To develop a comprehensive automated suite of tests (Unit Tests) for verifying the mathematical correctness of the isolated modules of Feedback and Rep Counter.
\end{enumerate}


\section{Extras}

\begin{enumerate}
    \item \textbf{Continuous Integration (CI):} GitHub Actions workflows to automate testing and code quality checks upon every commit.
    \item \textbf{Machine Learning Report (MLREPORT):} A comparative benchmark of different pose estimation backends (MoveNet, MediaPipe, HRNet) to evaluate trade-offs between inference latency and geometric accuracy, justifying the model selection.
\end{enumerate}

\begin{thebibliography}{9}

\bibitem{1}
Y. Zhang, M. Fuchs, and S. Mirhoseininejad, 
``FitCoachAR: Real-Time Adaptive Exercise Coaching via Pose Estimation and AR Feedback,'' 
CAS 772 Course Report, McMaster University, Jan. 2026.

\end{thebibliography}

\end{document}
