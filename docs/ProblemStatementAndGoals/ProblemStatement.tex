\documentclass{article}

\usepackage{tabularx}
\usepackage{booktabs}
\usepackage{hyperref}

\input{../Comments.text}
\input{../Common.text}

\title{CAS 741: Problem Statement and Goals\\\progname}

\author{\authname}

\date{\today}

\begin{document}

\maketitle

\begin{table}[hp]
\caption{Revision History} \label{TblRevisionHistory}
\begin{tabularx}{\textwidth}{llX}
\toprule
\textbf{Date} & \textbf{Developer(s)} & \textbf{Change}\\
\midrule
2026-01-17 & S. Mirhosseininejad & Initial Draft for CAS 741\\
2026-01-25 & S. Mirhosseininejad & Address Feedbacks: Clarify FitCoachAR purpose, fix LaTeX quotes, refine scope\\
\bottomrule
\end{tabularx}
\end{table}

\section{Problem Statement}
Musculoskeletal injuries during resistance training are frequently caused by improper exercise technique. While personal coaching is effective for injury prevention, it is often inaccessible or expensive. Existing mobile solutions for automated feedback lack rigorous biomechanical validation and often fail to provide actionable correction.

The project, \textbf{FitCoachAR}, addresses this gap by providing an augmented reality framework for movement analysis. The core problem is to develop a scientifically valid software pipeline that can ingest pose estimation data, apply geometric heuristics to detect deviations (e.g., knee valgus or spinal flexion), and provide feedback. The project leverages a prototype developed for mobile data analytics \cite{1}, refocusing it towards scientific rigor, modularity, and automated verification of the biomechanical algorithms.

\section{Goals}
\begin{enumerate}
    \item \textbf{Scientific Modularization:} Refactor the existing monolithic backend into decoupled, independently testable scientific computing modules:
    \begin{itemize}
        \item \textbf{Pose Estimator:} Handles the interface with computer vision models (MoveNet/MediaPipe) and geometric coordinate normalization.
        \item \textbf{Rep Counter:} Manages the state machine for cycle detection and phase tracking based on angular velocity.
        \item \textbf{Feedback Engine:} Executes the ``FormScript'' logic to generate corrective guidance based on biomechanical thresholds.
    \end{itemize}
    \item \textbf{Data Persistence \& Analysis:} Integrate a relational database (SQLite) to store user profiles and frame-by-frame biomechanical data, enabling post-session statistical analysis of movement quality.
    \item \textbf{Validation \& Verification:} Develop a comprehensive automated test suite (Unit Tests) to verify the mathematical correctness of the isolated Feedback and Rep Counter modules, ensuring that the software reliably identifies form errors.
\end{enumerate}


\subsection{Inputs and Outputs}
\begin{itemize}
    \item \textbf{Inputs:} Video frames (live or pre-recorded); User-specific calibration parameters (Range of Motion).
    \item \textbf{Outputs:}
    \begin{itemize}
        \item Feedback Systems:
        \begin{itemize}
            \item \textbf{Intra-Rep Guidance:} Corrective feedback (e.g., ``Go deeper'') and AR directional arrows that are provided \textit{during} the phase of movement.
            \item \textbf{Post-Rep Summary:} Immediate, brief validation (e.g., ``Nice rep'' or ``Less swing'') which is triggered instantly upon the completion of repetition.
            \item \textbf{Post-Session Analysis:} A comprehensive report generated by LLM containing the overall ``Form Score'', classifications of frame-by-frame ``FormCode'', and the aggregated metrics of error.
        \end{itemize}
        \item Persistent records of the performance of user stored in a local database.
    \end{itemize}
\end{itemize}

\subsection{Stakeholders}
\begin{itemize}
    \item \textbf{Researchers/Developers:} Those who need a modular ``FormScript'' library for experimenting with the new algorithms of pose analysis.
    \item \textbf{Physiotherapists:} Those who need to define new exercise rules (FormCodes) without writing code.
    \item \textbf{End Users:} Those who need to track their improvement of form over weeks, which requires the persistence of data.
\end{itemize}

\subsection{Environment}
The system is using Python for the logic of backend. Also, the core logic relies on \textbf{NumPy} for the calculations of vectors and \textbf{MediaPipe/MoveNet} for the data of pose. The new iteration will be introducing \textbf{SQLite/SQLAlchemy} for the persistence of data and Pytest for the verification purpose.


\section{Extras}

\begin{enumerate}
    \item \textbf{Continuous Integration (CI):} GitHub Actions workflows to automate testing and code quality checks upon every commit.
    \item \textbf{Machine Learning Report (MLREPORT):} A comparative benchmark of different pose estimation backends (MoveNet, MediaPipe, HRNet) to evaluate trade-offs between inference latency and geometric accuracy, justifying the model selection.
\end{enumerate}

\begin{thebibliography}{9}

\bibitem{1}
Y. Zhang, M. Fuchs, and S. Mirhoseininejad, 
``FitCoachAR: Real-Time Adaptive Exercise Coaching via Pose Estimation and AR Feedback,'' 
CAS 772 Course Report, McMaster University, Jan. 2026.

\end{thebibliography}

\end{document}
